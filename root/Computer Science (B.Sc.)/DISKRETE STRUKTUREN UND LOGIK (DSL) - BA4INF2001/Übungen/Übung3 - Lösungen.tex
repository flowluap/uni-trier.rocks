

\documentclass[11pt]{scrartcl}
%\usepackage {amsmath}
\usepackage {amssymb}
\usepackage[utf8]{inputenc}

\title{Lösung Übung DSL Ü3 Semester 1}
\author{Paul Wolf}


\begin{document}
\maketitle
\tableofcontents

\section{A1}
\label{sec:A1}

$X_k=\{1,\dots,k\}$
\begin{flushleft}
$X_5\setminus X_4 \cup X_3\setminus X_2 \cup X_1 = \{1,3,5\}$
\end{flushleft}
\begin{flushleft}
	$X_6\setminus X_5 \cup X_4\setminus X_3 \cup X_2 \cup X_1 = \{2,4,6\}$
\end{flushleft}
\begin{flushleft}
	$X_n\setminus X_n-1 \cup X_2\setminus X_1 = \{4,\dots,n\}$
\end{flushleft}

\section{A2}
\label{sec:A2}

$O_z = \varnothing$
$n_z' = n_z \cup \{n_z\}$

\subsection*{Behauptung:}
$n_z \subseteq 2^n_z$

\subsection*{Beweis:}
Ja: $O_z \subseteq 2^0_z$
$\varnothing \subseteq 2^\varnothing_z$
\begin{flushleft}
$\{1\} \nsubseteq 2^{\{1}\}=\{\varnothing,\{1\}\}$
\end{flushleft}

\subsection*{IV:} $n_z \subseteq 2^{n_z}$

\subsection*{IB:} $n_z' \subseteq 2^{n_z}$

\subsection*{IS:} $n_z' = n_z \cup \{n_z\} z$
					$\subseteq 2^{n_z} \cup \{n_z\}$
					$\subseteq 2^{n_z} \cup \{n_z\}$
					$=2^{n_z'}$
\section{A3}
\label{sec:A3}
\subsection{a)}
$(A \mid B)= A\setminus (A \cap \overline{B})$
$ = A\cap  \overline{A\cap B}$
$= A\cap (\overline{A} \cup \overline{B})$
$= A\cap (\overline{A} \cup B)$
$= (A \cap \overline{A}) \cup (A \cap B)$
$= \varnothing \cup (A \cap B)= A \cap B$

\subsection{b)}
$(A \cap B) \setminus (A \cap B)=$
$A \cap B \cap (\overline{A \cap C})=$
$A \cap B \cap (\overline{A} \cap \overline{C}) =$
$(A \cap B \cap \overline{A}) \cup (A \cap B \cap \overline{C})$
\begin{flushleft}
$(A \cap B) \cap \overline{C} = (A \cap B) \setminus C$
\end{flushleft}

\subsection{c) $(A \triangle B) \cap C = (A \cap C) \triangle (B \cap C$)}

\subsection*{"$\subseteq$"  Sei $x \in A \triangle B \cap C$, d.h. $x \in A \triangle B$ und $x \in C$ zwei Fälle für $x \in A \triangle B$}


\subsubsection{i}
$x \in A \setminus B$,$x \in C$
$\Rightarrow x \in A$,$x \not\subset B$,$x \in C$
$\Rightarrow x \in A \cap C$,$x \not\subset B$
$\Rightarrow x \in (A \cap C) \setminus B$
$\Rightarrow x \in (A \cap B) \setminus (B \cap C) \subseteq (A \cap C) \triangle (B \cap C)$

\subsubsection{ii}
$x \in B \setminus A$,$x \in C$ Analog wie i) folgt $x \in (B \cap C) \triangle (A \cap C)$ Also $x \in (A \cap C) \triangle (B \cap C)$ da man Operanten für "A" vertauschen darf
$\curvearrowright (A \triangle B) \cap C \subseteq (A \cap C) \triangle (B \cap C)$

\subsubsection*{"$\supseteq$" Sei $x \in (A \cup C) \triangle (B \cup C)$ d.h.}
 
\subsubsection{i}
$x \in A \cap C$,$x \not\subset B \cap C$
$\Rightarrow x \in A$,$x \in C$
Außerdem folglt $x \not\subset B$, da andernfalls $x \not\subset C$ mit $A_n x \not\subset B \cap C$
$\Rightarrow x \in A \setminus B \cap C \subseteq (A \triangle B) \cap C$
\subsubsection{ii}
$x \in B \cap C$,$x \not\subset A \cap C$
Analog zu Fall i) folgt $x \in (B \triangle A) \cap C$
$\Rightarrow x \in (A \triangle B) \cap C$ wie vorher auch.
$\curvearrowright (A \cap C) \triangle (A \cap B) \subseteq A \triangle B \cap C$

\end{document}
