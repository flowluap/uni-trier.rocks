

\documentclass[11pt]{scrartcl}
%\usepackage {amsmath}
\usepackage {amssymb}
\usepackage[utf8]{inputenc}

\title{Lehramt Mathe Vorlesung 4 Semester 1 !Achtung noch unvollständig!}
\author{Paul Wolf}


\begin{document}
\maketitle
\tableofcontents

\section{Vorab}
\label{sec:vorab-1.2.14}

Prinzip der vollständigen Induktion:
\begin{itemize}
\item Eine Aussage A(1) ist richtig (Induktionsanfang IA)
\item Aus A(n) folgt A(n-1) (Induktionsschritt IS)(Oder aus $A(1)\dots A(n)$ folgt $A(n-1)$)
\end{itemize}
Manchmal will man Aussagen $ A(n), n \in \mathbb{Z} , z \geq \mathbb{N} $ zeigen. Dann muss man $A(n)$ als wahr nachweisen und dann zeigen, dass aus $A(n)$ wieder $A(n+1) $ für $n \geq \mathbb{N}$ folgt. Ein Fall für die vollständige Induktion ist:


\section{1.2.14 Satz:}
\label{sec:satz-1.2.14}

Für jedes $n \in \mathbb{N}$ gilt:

$1+2+\dots+n = \sum \limits_{\nu=1}^n \nu = \frac{n(n+1)}{2} $

\subsection*{Beweis (vollständige Induktion):}

\begin{enumerate}
\item Induktionsanfang (IA):

$n=1$,
$\sum \limits_{ \nu}^n \nu = 1 = \frac{1(1+1)}{2}$

\item Induktionsschritt (IS):

$\sum \limits_{ \nu}^n \nu = \frac{1(1+1)}{2}$ (Induktionsannahme)

z.z $\sum \limits_{ \nu}^n+1 \nu = \frac{(n+1)(n+1-1)}{2}$

\item Es gilt:

$\sum \limits_{ \nu}^n+1 \nu = n+1+\sum \limits_{ \nu}^n = (n+1) + \frac{(n)(n+1)}{2} = \frac{2(n+1)+n(n+1)}{2} = \frac{(n+1)(2+n)}{2}= \frac{(n+1)(n-1+1)}{2} $


\end{enumerate}

\section{1.2.15 Definition:}

Für $n \in \mathbb{N_{O}}$ gilt:
\begin{flushleft}
$0! :=1$
\end{flushleft}
\begin{flushleft}
$n! := 1*2*\dots = \prod \limits_{\nu=1}^{n} \nu $ , $ n \geq 1 $
\end{flushleft}
\begin{flushleft}
$0! =1;1!=1;2!=2;3!=6;4!=24$ etc.
\end{flushleft}

\section{1.2.16 Satz:}
Ist $n \in \mathbb{N}$, dann hat die Menge $\prod_{n} := S_{n} := \sigma_n := \{\pi:\{1,n\} \rightarrow \{1,\dots,n\}:\pi$ bijektiv$\}$ $n!$ Elemente. $ S_{n}$ heißt auch meist Permutationsgruppe/Geometrische Gruppe.

\section{1.2.17 Definitionen: (Binomialkoeffizienten)}
Es seien $n, \nu \in \mathbb{N_{O}}$.

\begin{flushleft}
$ {n \choose 0} := 1 $ (gelesen: n über 0)
\end{flushleft}
\begin{flushleft}
$ {n \choose \nu} := \frac{n(n-1 \dots (n-\nu +1))}{1*2*\dots*\nu} =
\prod \limits_{k=1}^{\nu} \frac{n-k-1}{k} $ (für $\nu \geq 1$ gelesen Enn über Nü)
\end{flushleft}















\section{1.2.19 Satz: (Rekursionsformel für die Binomialkoeffiziente)}

Ist $n \in \mathbb{N_{O}}$,$\nu \in \mathbb{N}$, dann gilt:
\begin{flushleft}
	$ {n \choose \nu -1} +{n \choose \nu} = {n+1 \choose \nu} $ 
\end{flushleft}

\subsection*{Beweis:}
\begin{enumerate}
	\item Fall $1 \leqslant \nu \leqslant n$:
	Dann gilt nach 1.2.18, dass
	${n \choose \nu -1} + {n \choose \nu} = 
	{\frac{n!}{(\nu -1)! (n-1-\nu)!}}+{\frac{n!}{\nu!(n+1)!}} = 
	{\frac{n!\nu}{\nu!(n+1-\nu)!}}+{\frac{n!(n+1-\nu)}{\nu!(n+1-\nu)!}}=
	{\frac{n!(\nu+n+1-\nu)}{\nu!(n+1-\nu)!}}=
	{\frac{n!(n+1)}{\nu!(n+1-\nu)!}}={n+1\choose \nu}
	$
	\item Fall $\nu = n+1$:
	${n \choose \nu-1 }+{n \choose \nu}=0={n+1 \choose \nu }$
	
\end{enumerate}

\section{1.2.20 Definition:}
Ist A eine endliche Menge, so sei
$|A| := \#A$ die Anzahl der Elemente von A. Ist A nicht endlich, so schreibt man $|A| := \#A=+\infty$

\section{1.2.21 Satz:}
Es sei $n \in \mathbb{N}$, dann gilt für alle $\nu \in \mathbb{N_{O}}$, dass 
\begin{flushleft}
	$ |\{M \subset \{1,\dots,n\}: |M|= \nu \}| = {n \choose \nu}$ 
\end{flushleft}
Die Anzahl der k-elementigen Teilmenge einer n-elementigen Menge ist ${n \choose \nu}$
\subsection*{Beispiel ($n=3,\nu=2$):}
$|\{M \subset \{1,2,3\}:|M|>2\}|=|\{1,2\},\{1,3\},\{2,3\}|=3$
\begin{flushleft}
	${n \choose 3} = \frac{3!}{2!(3-2)}=\frac{6}{2*1}=3$
\end{flushleft}
Insbesondere ist ${n \choose k} \in \mathbb{N_{O}}$ für alle $n,\nu \in \mathbb{N_{O}} $
\subsection*{Beweis}






\end{document}
