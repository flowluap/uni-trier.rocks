

\documentclass[11pt]{scrartcl}
%\usepackage {amsmath}
\usepackage {amssymb}
\usepackage[utf8]{inputenc}

\title{Lösung Übung DSL Ü4 Semester 1}
\author{Paul Wolf}


\begin{document}
\maketitle
\tableofcontents
\section*{}
(sorry für das Deutsch, habe es nur abgeschrieben...) \\
Im Folgenden werden teilweise Worte, wie "und", "oder" durch ihre logischen Operanten $\land$ , $\lor$ ersetzt.

\section{A1}
E seien $A,B,C,D \subseteq U$ Mengen. Zeigen Sie: \\
\subsection{1}
$(A \cap B )x(C \cap D) = (AxC) \cap (BxD)$ \\
Recall: $AxB = \{(x,y) \mid x \in A, y \in B\}$ \\
$(AxB)x(C \cap D) = \{(x,y)\mid x \in A, y \in B \land y \in C \cap D\}$\\
$= \{(x,y)\mid x \in A \land x \in B \land y \in C \land y \in D\}$\\
$= \{(x,y)\mid (x,y) \in AxC \land (x,y) \in BxD \}$\\
$= (AxC) \cap(BxD)$

\subsection{2}
$(A \cup B)x(C \cup D) = \{(x,y) \mid x \in A \lor x \in B, y \in C \lor y \in D \}$ \\
$= \{(x,y) \mid x \in A, y \in C $ oder \\
$x \in A, y \in D $ oder \\
$x \in B, y \in C $ oder \\
$x \in B,y \in D \}$

$= \{(x,y) \mid (x,y \in AxC) \in AxD \\
(x,y) \in BxC \\
(x,y) \in BxD \}$ \\
$= (AxC) \cup (AxD) \cup (BxC) \cup (BxD)$

\subsection{3}
$(\overline{AxB})=(\overline{A} xB) \cup (Ax \overline{B}) \cup (\overline{A} x \overline{B})$ \\
1: $\overline{A}^u$(u ist das Universum) \\
2: $\overline{B}^u$ \\
Recall: $\overline{A} = \{ x \mid x \notin A, x \in U \}$ \\
$\overline{A} = U \setminus A$

\subsubsection{i}
Sei $(x,y) \in \overline{AxB}$ \\
$\Rightarrow x \in \overline{A}, y \in B $oder \\
$x\in \overline{A}, y \in \overline{B}$ oder \\
$x \in A, y \in \overline{B}$ \\
$\Rightarrow (x,y) \in \overline{A} x B$ oder \\
$(x,y) \in \overline{A} x \overline{B}$ oder \\
$(x,y) \in A x \overline{B}$
$\Rightarrow (x,y) \in (\overline{A} x B) \cup (A x  \overline{B}) \cup (\overline{A} x \overline{B})$
\subsubsection{ii}
Sei $(x,y) \in (\overline{A} x B) \cup (A x \overline{B}) \cup (\overline{A} x \overline{B})$ \\
$\Rightarrow genau umgekehrt (i)$

\subsection{4}
$(AxB) \setminus (CxD) = (Ax(B \setminus B)) \cup ((A \setminus C) xB)$ \\
$(AxB) \setminus (CxD) = (AxB) \cap (CxD)$ \\
$= (AxB) \cap ((\overline{C} xD) \cup (Cx \overline{D}) \cup (\overline{C} x \overline{D})) $ \\
$= (AxB) \cap (\overline{C} x D) \cup (AxB) \cap (C x \overline{D}) \cup (AxB) \cap (\overline{C} x \overline{D})$ \\
$= (A \cap \overline{C})x(B \cap D) \cup (A \cap C)x(B \cap \overline{D}) \cup (A \cap \overline{C})x(B \cap \overline{D})$ \\
$Ax(B \cap \overline{D}) \cup ((A \cap \overline{C})xB)$ \\
$= (Ax(B \setminus D)) \cup ((A \setminus C)xB)$
\subsubsection*{Anmerkung}
$(AxB) \cup (CxD) \neq (A \cup C)x(B \cup D)$ \\
$(AxB) \cup (CxD) \subseteq (A \cup C)x (B \cup D)$ \\
Wenn (A1a) die Vereinigung nicht stimmt...

\section{A2}
1. Familie, Freunde \\
$(2,3)$ sei $R \subseteq X = X$ eine Relation. \\
Dann
\subsection*{i}
R ist reflexiv $iff \overline{R} \subseteq XxX$ ist irreflexiv \\
\subsection*{ii}
R ist irreflexiv $iff \overline{R} \subseteq XxX$ ist reflexiv \\

Beweis sei $R \subseteq XxX$ eine Reflexion \\
dann $\forall x \in X : (x,x) \in R$ \\
durch Definition von $\overline{R} : (x,y) \in R \Rightarrow (x,y) \notin \overline{R}$ \\
das gilt auch für $(x,y)$ und so $\forall x \in X : (x,x) \notin R$ \\
Deshalb ist $\overline{R}$ irreflexiv. \\

$III^{log}$ (genauso ähnlich):\\
$\forall x \in X : (x,x) \notin R \Rightarrow \neg (x,y) \notin \overline{R}$ \\
durch doppelte negation: $(x,y) \in \overline{R}$ \\
Der umgekehrte Teil folgt von dem Fakt, dass $\overline{(\overline{R})} = R$ durch relativ Komplement vom relativen Komplement. \\

\subsection*{}
$(2,3)$ sei $R \subseteq X = X$ eine Relation. Dann ist R symmetrisch $\iff \overline{R} \subseteq XxX$ auch symmetrisch: \\
\subsection{Beweis}
Sei $R \subseteq XxX$ ist symmetrisch. \\
Durch symmetrie von Relationen ist symmetrisch \\
$(x,y) \in R \iff (x,y) \in R$ \\
Annahme, dass $\overline{R} \subseteq XxX$ ist antisymmetrisch, dann \\ 
$\exists (x,y) \in \overline{R} : (y,x) \in \overline{R}$ \\
Aber durch Def. von $\overline{R} (y,x) \in R$ \\
$R ist symmetrisch (x,y) \in R \Rightarrow \Leftarrow$ zu $(x,y) \in \overline{R}$ \\
So ist $\overline{R}$ symmetrisch. Ähnlich folgt die umgekehrte Richtung.

\subsection{Anmerkung}
$(x,y) \in R \iff (y \in X) \in R$ \\
Beweis: Sei R symmetrisch \\
$(x,y) \in R \Rightarrow (y,x) \in R$ \\
$(y,x) \in R \Rightarrow (x,y) \in R$ \\
$(x,y) \in R \iff (y,x) \in R$ (bidirektional)

\section{A3}
Die Kugel (a,c) und (b,c) müssen im Komplement der geschnittenen Menge liegen d.h. in $(R \cup R^{-})$. Der Fall $(a,c) \in R$ und $(b,c) \in R^{-}$ würde wegen der Transitivität $(a,b) \in R$ ergeben, was der Vorraussetzung wiederspricht, analog kann $(a,c) \in \overline{R}$ und $(b,c) \in R^{-}$ nicht sein, da auch $R^{-}$ transitiv ist und dies $(a,b) \in R^{-}$ ergeben würde. \\
Die Behautung wird nur erfüllt, wenn beide Tupel entweder in $R$, oder $\overline{R}$ liegen.

\section{A4}
$M = \{0,1,2,3,4,5\}$ \\
$R,S \subseteq MxM$ \\
$R = \{(1,1),(2,1),(3,1) \}$ \\
$S = \{(1,1),(1,3),(3,3),(4,5) \}$ \\
$RoS = \{ (1,1),(1,3),(2,1),(2,3),(3,1),(3,3) \} $ \\
Extra: $SoR = \{(1,1),(3,1) \}$ \\

\subsection*{Definition}
$RoS = \{ (x,z) \mid \exists y \in Y, (x,y) \in R \land (y,z) \in S\}$ \\
$R \cap S = \{ (x,y) \mid (x,y) \in R \land (x,y) \in S \}$ \\
$R \cup S = \{ (x,y) \mid (x,y) \in R \lor (x,y) \in S \}$



\end{document}
